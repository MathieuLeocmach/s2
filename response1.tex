\documentclass[a4paper, rebuttal, parskip=true, firsthead=false, fromemail=true, foldmarks=false]{scrlttr2}
\usepackage{amsmath}
%\usepackage{amsfonts}
%\usepackage{amssymb}
\usepackage[british]{babel}
\usepackage{hyperref}
\usepackage[binary]{SIunits}
\begin{document} 
\begin{letter}{Dr. James Skinner, Associate Editor
The Journal of Chemical Physics
University of Wisconsin
editor-jcp@chem.upenn.edu }
\opening{\bf Dear James,}

Thank you very much for your e-mail concerning our manuscript (GLATRA12.10.0133) together with the comments of the Reviewers.

Following the constructive comments and suggestions of the Reviewers, we have revised our manuscript. 
We believe that we have been able to answer the comments of both reviewers on a satisfactory level. We note that the parts revised to answer the Reviewers have been highlighted with blue characters in the revised manuscript. 


We hope that you and your reviewers would find that the revised manuscript is now suitable for publication in the Journal of Chemical Physics. 

\closing{\bf Sincerely yours,} 
\clearpage

\textsf{\textbf{Replies to the comments of Referee 1}}

First, we thank the Reviewer for having carefully read our revised manuscript and provided useful comments to improve our manuscript. 
Hereafter we reply to the comments one by one.

\begin{quotationi}
The manuscript by Leocmach, Russo and Tanaka is very interesting, well-conceived, with a large amount of original and novel ideas and concepts. Moreover, it fits extremely well to a special Issue for the Glass Transition, because it studies in detail the crystal/glass interplay in polydisperse hard spheres. While hard spheres are one of the most studied models for glass transition in colloidal experiments, most simulations deal with either monodisperse hard spheres or with binary systems, while real systems are always polydisperse. Therefore, the present work helps to elucidate several concepts also based on the microscopic understanding of the role of polydispersity, an issue that has often been overlooked. The results are sound and challenging, as they call for the importance of many-body correlations to describe the occurrence of the glass transition, while most theories rely on pair correlation functions. Therefore, I strongly recommend that the paper is published in J. Chem.
Phys, after the authors address the following (minor) points.


Page 1

To state that MCT is the only exact theory of the glass transition (in the mean-field, infinite dimension approximation) is not correct. There is an ongoing debate whether this is true or not, and the fact that the infinite-d limit of MCT does not coincide with that of RFOT and replica theory makes the case quite controversial. I would remove the word ``exact''.
\end{quotationi}

\begin{quotationi}
Page 2

A clarification concerning the calculation of the particle free volume: what does it mean that all faces of the Voronoi polyhedron ``are shifted normally''? If this is a standard procedure, then a reference should be added.
Please explain the choice Nb(i)=12. Are these the 12 closest neighbours?
\end{quotationi}

\begin{quotationi}
Page 3

A few more words to explain the two crucial observables $C$ and $Q_6$ should be added, to avoid that the reader has to understand their meaning (compared to other observables) without having to read References 28 and 29.
\end{quotationi}

\begin{quotationi}
Section A

Fig. 2 is not very intuitive. There is a lot of information which is not clear. For example, systematically the more ordered configurations are not the most probable. But the text reads ``the order parameters are maximized or minimized by different local structures'': superficially one would go to look at the maxima or minima in the figure, while instead it is the value of the observable (not its probability) to be maximized or minimized. Perhaps, it would help to add some cartoons or labels to associate each observable with what it measures. The mobilities are also difficult to understand. I suggest that the authors to try to improve the figure, also because this is an early figure and it is propedeutic for the rest of the paper.
\end{quotationi}

We tried to improve the description of the figure to lift the ambiguity between a structure maximizing the parameter and a structure with maximum probability. We also added on each subfigure the nature of the most ordered state. We hope this helps to make the figure more propedeutic.

\begin{quotationi}
Section B

It is not clear to me why the calculation of $S_x$ has been limited to a 10\% subset of particles. The authors claim that there is no significant dependence on this, since they have changed the threshold value. What whould happen if $S_x$ is calculated for all particles? How many configurations (with 10\% particles) roughly do they need to average in order to obtain the curves of Fig.4?
\end{quotationi}

What we meant is that changing the threshold to include only 5\% or 25\% did not change qualitatively the results. The more we include particles, the closer we get to the trivial density structure factor $S(q)$, and the more difficult it will be to extract a length scale. A low threshold do not discriminate between ordered and unordered particles. Indeed, around 30\% direct visualization of the patterns like in our Fig. 6 becomes difficult and over 50\% $S_x$ is basically flat at low $q$. 

At the other end of the spectrum, including too few particles decreases the number of pairs very quickly ($N_\text{pairs}\sim N^2$) and thus the signal to noise ratio become very poor. Indeed at 0\% there is no signal anymore.

The good threshold is a balance between these two extremes, and we found that between 5\% and 40\% the numerical results were consistent, with an easier direct visualization around 10\%.

For each pressure, we ran 8 seeds for a few $\tau_\alpha$ (after equilibration) summing up to approximately $10^4$ configurations. The correlation lengths are already tractable although noisy with 10 times less configurations.

\begin{quotationi}
Section C

A couple of curiosities on the role of polydispersity.

1) Can the authors associate anything special happening at Delta $\sim6-7\%$ where (one-phase) crystallization according to Fasolo and Sollich is suppressed? For example, repeating the plot in Fig.10 at higher pressures, could reproduce (and provide a microscopic insight) these results?
\end{quotationi}

\begin{quotationi}
2) How does the results presented here connect to those of MD simulations of polydisperse hard spheres by Zaccarelli et al (PRL 2009) that the dynamics is not much affected by changing Delta (for small values of the polydispersity)?
\end{quotationi}

\begin{quotationi}
3) Any ideas on why the correlation length seems to saturate at Delta~13-15\%?
\end{quotationi}

\begin{quotationi}
Fig.8 : The colors should be adjusted, as when printed in black and white nothing can be distinguished.
\end{quotationi}
We changed the figure to gray scale to allow black and white printing.

\begin{quotationi}
In the text description and in the labels of Fig. 10 add the word ``fluid'' associated to the data of Delta=0 and 4\% in order to facilitate the reader.
\end{quotationi}

We hope that the Reviewer would think that the revised manuscript is now suitable for publication in the Journal of Chemical Physics. 


\textsf{\textbf{Replies to the comments of Referee 2}}

First, we thank the Reviewer for having carefully read our revised manuscript and provided useful comments to improve our manuscript. 
Hereafter we reply to the comments one by one.

\begin{quotationi}
This papers reports on new computer simulations for the glass transition
in polydisperse hard sphere systems. A particular focus is on a growing
length scale during glass formation. It is shown that many-body
correlations exhibit such a growing length scale while such a length scale
is absent for pair
correlations, one of the key figures in Figure 5.
This is of basic consequence as in standard mode
coupling theories only pair correlations enter.
By bond orientation
order parameters which are sensible for icosaedral ordering the competiion
between amorpjization and crystallization is highlighted.

The paper contains many novel results and shows interesting plots of
distributions and correlations which shed new light on the nature of
the glass transition itself. The paper is clearly written
and can make a large impact on future studies.

I recommend publication in JCP but the authors should amend the paper
according to the following points:

1) Recently the crystallinity criteria based on q6 and q4 distributions
were criticized, see Kapfer et al, PRE 85, 030301 (2012). Can the authors
comment on the validity of their criteria in the light of this recent paper?
\end{quotationi}

\begin{quotationi}
2) paper 2, chapter II: a) if equation (1) is an entropy, it needs unit kT.
b) "The pair correlation function of each particle": can the authors
define precisely what they mean by this strange object?
\end{quotationi}

\begin{quotationi}
3) page 4: it is stated that a more reliable procedure to obtain the
length scale is in Fourier space. This sounds odd to me, as position and
Fourier space provide in principal the same information.
\end{quotationi}

We hope that the Reviewer would think that the revised manuscript is now suitable for publication in the Journal of Chemical Physics.



%\cc{Cclist} 
%\ps{adding a postscript} 
%\encl{list of enclosed material} 
\end{letter} 
\end{document}