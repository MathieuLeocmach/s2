\documentclass[a4paper, rebuttal, parskip=true, firsthead=false, fromemail=false, foldmarks=false]{scrlttr2}
\usepackage{amsmath}
%\usepackage{amsfonts}
\usepackage{amssymb}
\usepackage[british]{babel}
\usepackage{hyperref}
\usepackage[binary,squaren]{SIunits}
\begin{document} 
\begin{letter}{Dr. James Skinner,\\Associate Editor
The Journal of Chemical Physics
University of Wisconsin\\
editor-jcp@chem.upenn.edu }
\opening{Dear Editor,}

Thank you for sending us two Referee reports for our manuscript GLATRA12.10.0133 entitled
\emph{Importance of many-body correlations in glass transition: an example from polydisperse hard spheres}.

We were very pleased to read that both Referees have appreciated our work and provided constructive comments to
improve its clarity and impact. Following these suggestions we have thus revised our manuscript, and all the changes
are detailed in the following response to the Referees.

We hope that you will find the revised manuscript suitable for publication in the Journal of Chemical Physics. 

Sincerely yours,

\emph{Mathieu~Leocmach, John Russo  and Hajime~Tanaka}

\clearpage


\textsf{\textbf{Replies to the comments of Referee 1}}

We thank the Referee for appreciating our work and for providing useful comments to improve its clarity and impact.
To improve readability, we attach the original questions of the Referee and respond to them one by one.

\begin{quotationi}
[...] I strongly recommend that the paper is published in J. Chem. Phys, after the authors address the following (minor) points.
\newline\newline
Page 1
To state that MCT is the only exact theory of the glass transition (in the mean-field, infinite dimension approximation) is not correct. There is an ongoing debate whether this is true or not, and the fact that the infinite-d limit of MCT does not coincide with that of RFOT and replica theory makes the case quite controversial. I would remove the word ``exact''.
\end{quotationi}

We thank the Referee for pointing out that indeed the debate over which is the exact mean-field theory of the glass transition is still a matter of intense research.
To avoid confusion we decided to fully follow the Referee suggestion and rewrite the previous sentence in the following form.

{\it Page 1, second column, end of first paragraph:\\
This question is particularly important considering that the Mode-Coupling Theory (MCT) of the glass transition takes as input only two-body quantities, and
similarly modern spin-glass-type theories of the structural glass transition~[10-12] are not taking explicitly into account many-body correlations.
}


\begin{quotationi}
Page 2

A clarification concerning the calculation of the particle free volume: what does it mean that all faces of the Voronoi polyhedron ``are shifted normally''? If this is a standard procedure, then a reference should be added.
Please explain the choice Nb(i)=12. Are these the 12 closest neighbours?
\end{quotationi}

The calculation of the free volume for each particle is a two-step procedure. First the system volume $V$ is univocally partitioned between all particles,
such that each particle $i$ is assigned a volume $V_i$, and $\sum_i V_i=V$. The free volume $v(i)$ is the subset of $V_i$ in which the excluded volume of
the particle can move without leaving $V_i$. The most popular way of dividing space into non-overlapping regions is the Voronoi decomposition, which assigns to each
particle a polyhedron whose faces lie normally to vectors connecting the center of the particle with that of its neighbours. The free volume $v_i$ is
then the volume of the polyhedron obtained by radially shifting all faces of the Voronoi polyhedron $V_i$ of the quantity $d_i/2$, where $d_i$ is the diameter of particle $i$. In this way the volume $v(i)$ represents the volume in which the particle can move without leaving $V_i$.

The choice $N_b$ indeed corresponds to the $12$ closest neighbours. Neighbours identification is usually done by considering all particles within
a certain distance from the central one. This has two main disadvantages: 1) the distance between particles fluctuates and with a fixed cutoff the
identity of neighbours is strongly affected by these fluctuations 2) the cutoff distance should be tailored for each pressure.
When looking for crystalline particles, we instead choose to focus on the $12$ closest neighbours, because all the relevant crystal structures for
hard-spheres have $12$ nearest neighbours. This criteria is indepentent of pressure and we have shown (Ref.~[28] and [30] in the manuscript) that it considerably
improves the strength of the crystal identification procedure.

To clarify these points in the manuscript we have made the following changes

{\it Page 2, second column, first paragraph:\\
To compute the free volume $v(i)$ we follow previous
studies: first the Voronoi-diagram for each configuration is computed, and the polyhedron
surrounding each particle is determined. To account for polydispersity we employ the radical Voronoi tessellation.
The free volume of particle $i$ is then
computed by shifting normally all the faces of the corresponding polyhedron by $\sigma(i)/2$
towards particle $i$, and computing the new volume. In this way
the volume $v(i)$ represents the volume in which the excluded volume of particle $i$ can move without leaving its Voronoi cell.
This procedure is conducted independently
for each particle and for each configuration.
}

{\it Page 2, second column, second paragraph:\\
Instead we
sort neighbours according to their distance from particle $i$, and
fix $N_b(i)=12$ which is the number of nearest neighbors in icosahedra and close packed crystals (like \textsc{hcp} and \textsc{fcc})
which are known to be the only relevant crystalline structures for hard spheres.
}


\begin{quotationi}
Page 3

A few more words to explain the two crucial observables $C$ and $Q_6$ should be added, to avoid that the reader has to understand their meaning (compared to other observables) without having to read References 28 and 29.
\end{quotationi}

We agree with the Referee that the role of the order parameters $\text{C}$ and $Q_6$ should be better explained in the manuscript. To do so we added the following
paragraph

{\it Page 3, first column, paragraph before Results:\\
The crystallization transition is characterized by
the symmetry breaking of both orientational and translational order.
We note that while both $\text{C}$ and $Q_6$ are good measures of bond-orientational order, the density or other two-body
quantities are measures of translational order.  It was shown~[28] that hard spheres crystallization is
driven by fluctuations in bond-orientational order and not by density fluctuations. Crystals continously form, grow and melt
in regions of high bond-orientational order, which then effectively act as precursors for the crystallization transition.
So $\text{C}$ and $Q_6$, while not
being direct indicators for the presence of crystals, rather measure the tendency to promote crystallization. In Section~IIIB we are going to show
that indeed the lengthscale associated with bond-orientational order fluctuations increases with supercooling. Then in Section~IIIC we are going to study
the mechanism by which the crystallization transition is avoided.
}

\begin{quotationi}
Section A

Fig. 2 is not very intuitive. There is a lot of information which is not clear. For example, systematically the more ordered configurations are not the most probable. But the text reads ``the order parameters are maximized or minimized by different local structures'': superficially one would go to look at the maxima or minima in the figure, while instead it is the value of the observable (not its probability) to be maximized or minimized. Perhaps, it would help to add some cartoons or labels to associate each observable with what it measures. The mobilities are also difficult to understand. I suggest that the authors to try to improve the figure, also because this is an early figure and it is propedeutic for the rest of the paper.
\end{quotationi}

Following the Referee's suggestions we improved the readability of the Figure. To avoid the ambiguity between a structure maximizing the parameter and a structure with maximum probability, we have placed labels on the $x$-axis which indicate the structure which is targeted by a particular order parameter.
We also employed color on the $y$-axis to make visually very clear which curve refers to the probability distribution and which to the mobility.

We also changed the caption to make it more clear

{\it Caption of Figure 2:\\
Probability distributions (red dashed line) and mobility (blue continuous line) function of various order parameter at $\beta p\sigma^3=25$ and for a time difference corresponding to the $\alpha$-relaxation. Top row: two-body excess entropy $s_2$ (left) and pair free energy $f_2$ (right). Central row: local six-fold orientational orders $q_6$ (left) and $w_6$ (right). Bottom row: coarse grained $Q_6$ (left) and crystallinity $C$ (right). Mobility is in unit of mean-square displacement. The shaded area shows the contribution from $10\%$ of the particles having highest (for $f_2$, $q_6$, $Q_6$ and $C$) or lowest (for $s_2$, $w_6$) value of the order parameter.
}


\begin{quotationi}
Section B

It is not clear to me why the calculation of $S_x$ has been limited to a 10\% subset of particles. The authors claim that there is no significant dependence on this, since they have changed the threshold value. What whould happen if $S_x$ is calculated for all particles? How many configurations (with 10\% particles) roughly do they need to average in order to obtain the curves of Fig.4?
\end{quotationi}

What we meant is that changing the threshold to include only 5\% or 25\% did not change qualitatively the results. The more we include particles, the closer we get to the trivial density structure factor $S(q)$, and the more difficult it will be to extract a length scale. A low threshold do not discriminate between ordered and unordered particles. Indeed, around 30\% direct visualization of the patterns like in our Fig. 6 becomes difficult and over 50\% $S_x$ is basically flat at low $q$. 

At the other end of the spectrum, including too few particles decreases the number of pairs very quickly ($N_\text{pairs}\sim N^2$) and thus the signal to noise ratio become very poor. Indeed at 0\% there is no signal anymore.

The good threshold is a balance between these two extremes, and we found that between 5\% and 40\% the numerical results were consistent, with an easier direct visualization around 10\%.

For each pressure, we ran 8 seeds for a few $\tau_\alpha$ (after equilibration) summing up to approximately $10^4$ configurations. The correlation lengths are already tractable although noisy with 10 times less configurations.

We have added the following paragraph to the manuscript.

{\it Page 6, first column, 4th paragraph:\\
The choice of the threshold $x^*$ is a balance between taking in too many particles or too few. If too few (below $5\%$) $S_x$ is too noisy. If too many, the threshold does not discriminate between ordered and disordered particles and $S_x$ tends to the trivial density $S(q)$. We found that between $5\%$ and $40\%$ the absolute value of the length is marginally affected but not the pressure dependence. We chose to use $10\%$ across this paper because this value allows the easiest direct visualization on a single frame (Fig.~[6]). We checked the pressure independence of the two-body parameter's length with thresholds up to $90\%$.
} 

\begin{quotationi}
Section C

A couple of curiosities on the role of polydispersity.

1) Can the authors associate anything special happening at Delta $\sim6-7\%$ where (one-phase) crystallization according to Fasolo and Sollich is suppressed? For example, repeating the plot in Fig.10 at higher pressures, could reproduce (and provide a microscopic insight) these results?
\end{quotationi}

The loss of microscopic stability of the crystalline branch (i.e. the subset of particles which are identified as crystalline in the metastable melt)
is shown in Fig.10 as the fluid and crystalline branch loose their intersection. When the two branches don't cross anymore, orientational order
stabilizes icosahedral environments, and trajectories are not seen to crystallize. As pointed out by the Referee,
the loss of this crossover should occurr at higher values of polydispersity with
increasing pressure. We can confirm that at polydispersity $\Delta=7\%$ the two branches don't cross at any value of the pressure we examined, suggesting
that indeed one-phase crytallization is suppressed. But, differently from the low pressure, at high pressure we cannot examine
what happens at polydispersities below $\sim 6\%$, as the crystallization is not an activated process anymore
(this nucleation has been called spinodal-like, e.g. Valeriani et al., Soft Matter 8, 4960 (2012)), and trajectories crystallize
before entering any metastable supercooled state.

We have added the following sentence to the manuscript.

{\it Page 8, first column, before Discussion:\\
We also confirm that at polydispersity $\Delta=7\%$ icosahedral particles are favoured over crystalline ones for
all the pressures studied, in line with the observation of Fasolo and Sollich~[39], that
one-phase crystallization is suppressed at high polydispersity.
}



\begin{quotationi}
2) How does the results presented here connect to those of MD simulations of polydisperse hard spheres by Zaccarelli et al (PRL 2009) that the dynamics is not much affected by changing Delta (for small values of the polydispersity)?
\end{quotationi}

The results by Zaccarelli et al that the dynamics is not much affected by changing polydispersity are important since they tell
us that the differences between the results at different polydispersity are only due to structural properties, and not by
dynamic effects. For example, in Fig. 10 we compare results at different values of $\Delta$, but the results of Zaccarelli et al.
assure us that the differences between the trajectories (the monodispersed one crystallizing and the other not) is entirely due
to some structural mechanism, which is indeed what we propose.

We thank the Referee for bringing this point to the attention, and we have added the following sentence to the manuscript

{\it Page 8, first column, before Discussion:\\
Since the dynamics of the
system is approximetely the same at different values of $\Delta$~[37], this difference in the crystallization behaviour has to come from
some structural difference introduced by polydispersity, i.e. an increased population of icosahedral particles.}


\begin{quotationi}
3) Any ideas on why the correlation length seems to saturate at Delta~13-15\%?
\end{quotationi}

We believe that there will always be some degree of static correlation in a supercooled state, which
survives at the highest polydispersities we have studied. At even higher polydispersities, the system
will probably phase separate to counter-act the loss of structural order.


\begin{quotationi}
Fig.8 : The colors should be adjusted, as when printed in black and white nothing can be distinguished.
\end{quotationi}
We changed the figure to gray scale to allow black and white printing.

\begin{quotationi}
In the text description and in the labels of Fig. 10 add the word ``fluid'' associated to the data of Delta=0 and 4\% in order to facilitate the reader.
\end{quotationi}
We changed the figure according to the Referee's recommendation.

We hope that the Referee would think that the revised manuscript is now suitable for publication in the Journal of Chemical Physics. 

\clearpage

\textsf{\textbf{Replies to the comments of Referee 2}}

First, we thank the Reviewer for carefully reading our manuscript and providing useful comments for its improvement. 
Hereafter we reply to the comments one by one.

\begin{quotationi}

I recommend publication in JCP but the authors should amend the paper
according to the following points:

1) Recently the crystallinity criteria based on q6 and q4 distributions
were criticized, see Kapfer et al, PRE 85, 030301 (2012). Can the authors
comment on the validity of their criteria in the light of this recent paper?
\end{quotationi}

We thank the Referee for bringing to our attention a relevant work in the identification of
local crystallinity. In the work of Kapfer et al., the identification of crystalline
particles based uniquely on $q_6$ is criticized, as affected by spurious noncrystalline cells.
We also stand with the claim that $q_6$ alone cannot be used to distiguish crystalline particles. In the
paper by Kapfer et al., the proposed solution is the use of rank-four Minkowski tensors. In our work
we instead followed a more conventional approach, which is to get rid of the fluctuations which
cause the artifacts by spatially coarse-graining $q_6$ to form $Q_6$. The procedure was first
described in the Ref.~29 of our manuscript, and it has been confirmed to drastically improve the
quality of crystalline identification. Note also that the criteria adopted by Frenkel and co-workers
is also based on spatial coarse-graining, as it involves scalar products between $q_6$ tensors of
neighbouring particles.

In the paper we added the reference suggested by the Referee to this new method, and the following sentence:

{\it Page 3, first column, end of first paragraph:\\
Alternatively, we note that the shortcomings of
non coarse-grained order parameters in the identification of crystallinity can be addressed by Minkowski tensors~\cite{kapfer2012jammed}
}


\begin{quotationi}
2) paper 2, chapter II: a) if equation (1) is an entropy, it needs unit kT.
b) "The pair correlation function of each particle": can the authors
define precisely what they mean by this strange object?
\end{quotationi}

In the previous version of the manuscript we forgot to write that the unit of energy is
fixed so that the Boltzmann constant is $k_B=1$. In these units, equation 1 is correct, but now we include
$k_B$ to avoid confusion.

The pair correlation function $g_i(r)$ is defined as
$$
\rho g_i(r)=\langle \sum_{j\neq i}\delta(\mathbf{r}-\mathbf{r}_i+\mathbf{r}_j)\rangle
$$
where the average is not an ensamble average, but a temporal average over timescales much shorter
of the $\alpha$ relaxation time (for averages over long time scales we recover the usual definition
of the pair correlation function). This quantity is introduced to characterize the local environment
of each particle, and the short time average is done to get rid of high-frequency fluctuations ($\beta$ processes).
The calculation of $ g_i(r)$ is computationally expensive as it requires the calculation to be done at short
time intervals.

We have added the following sentence to clarify these points

{\it Page 2, first column, first paragraph of Methods section:\\
In the following we fix the unit of length as $\sigma_0=1$ and the unit
of energy so that the Boltzmann constant is unity, $k_B=1$.
}

{\it Page 2, first column, third paragraph of Methods section:\\
In practice, this requires time averages to compute the pair correlation function of each particle, $g_i(r)$,
where the particle distribution around a particle $i$ is averaged over short-time scales ($\beta$ processes).
}

\begin{quotationi}
3) page 4: it is stated that a more reliable procedure to obtain the
length scale is in Fourier space. This sounds odd to me, as position and
Fourier space provide in principal the same information.
\end{quotationi}

As the Referee points out, real space and Fourier space do carry the same information, however it is practically easier to extract long distance information from $S(q)$ than from $g(r)$ (conversely, short distance information is easier to extract from $g(r)$). The Referee may compare the Fourier space correlation function of our Figure 4 to the real space correlation function of the Figure 3d of Tanaka \emph{et al.} Nature materials 9, 324–31 (2010).

Real space correlation function is oscillatory when $r$ is larger than the diameter $\sigma$. Thus the correlation length is obtained by fitting the \emph{envelope} of the real space function. Since the function is rapidly decaying, noise becomes dominant for $r\gtrsim L/4$, which makes localization and amplitude of the peaks subject to interpretation. Since the Ornstein-Zernike expression is only asymptotic, the two first peaks should not be fitted. With our system size, this leaves only about 3-4 peaks, i.e. 3-4 points, to fit. To correct for the high noise levels of scalar order parameters, like $Q_6$, one can introduce tensorial order parameters. Indeed the Nature Material paper uses a tensorial order parameter $Q_{6m}$, effectively correlating 7 scalar fields, reducing the noise sevenfold, and thus allowing the fit of 5-6 peaks. But this procedure cannot be adopted for two-body quantities, like $f_2$, which have a purely scalar nature. So, to keep both two-body and many-body
correlation functions consistent, we plot only these correlations in Fourier space, where
the Ornstein-Zernike form at small $q$ is not oscillatory, and thus much easier to fit unambiguously (we can use 9 points for our fit).

We have added the following sentences to clarify this point:

{\it Pages 4-5, the paragraph in between:\\
The correlation lengths of crystal-like bond orientational order can be easily measured in real space due to the tensorial nature
of these order parameters, as was shown in previous studies~\cite{tanaka,kawasaki,mathieu_icosahedra,russo_gcm,russo_hs}. Two-body
correlation functions, both $f_2$ and $s_2$ being scalars, is more difficult to extract in real-space, especially at long distances,
where long averages are required to overcome the statistical noise. Theoretically real and Fourier space contain the same information, however real space correlation functions oscillate at large $r$ where their Fourier counterparts are not oscillatory at small $q$. To extract length scale from the former, one should fit the envelope of an oscillatory, rapidly decaying to noise level function, which is a difficult procedure. We explain in the following a straightforward procedure to extracting correlation length from Fourier space.
}



We hope that the Referee would think that the revised manuscript is now suitable for publication in the Journal of Chemical Physics.



%\cc{Cclist} 
%\ps{adding a postscript} 
%\encl{list of enclosed material} 
\end{letter} 
\end{document}